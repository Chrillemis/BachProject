
Our first exploration of the dataset, was taking a look at the correlation matrix in table \ref{tab:Correlation} and scatter matrix in figure \ref{fig:scattermatrix} We observe that vx and y are correlated and vy and x are correlated, both at present and at formation. This is due to the fact that the velocity vector is always perpendicular to the position vector in a circular orbit, and therefore when x or y is small, we get high vy and vx. We would get inverted signs for them in our correlation matrix if the orbits were reversed.

We see very strong correlations between the ratios of amounts of atoms in the star. Some of them are very collinear. For example if we look at [O/H] and [MG/H] we see an almost perfect linear relation. In general it seems like all of the scatter plots that plots H as baseline vs H as baseline are very collinear and similarly all scatteplots that plots Fe as baseline vs. Fe as baseline are very collinear. This leads us to believe that there really are only 2 independent variables hidden in all these concentrations i.e. the concentration of iron and the concentration of hydrogen.

We also see the age of the star is strongly correlated with the atomic-concentration fractions. We assume this is due to the star going through different phases of its life and thus changing its chemical composition. 
A similar thing is seen for mass and age where the older the star is the less mass it has. We assume this is due to the fact that stars in general lose mass over their lifetime as a consequence of evaporation and stellar winds, among other things. 
