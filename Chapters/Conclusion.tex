In conclusion, we have effectively creates models that are able to predict whether og not a star could be classified as a migrator. The best performing model was a pretty simpe neural network, consisting of a dense layer of 16 neurons, with a test accuracy of 88\%. This type of model, was the best performing of the models we ran on the sample data, which is why it was chosen to be trained on the full dataset.
The tree-based methods all performed well, achieving accuarcies around 87\% (depending on the specific model you choose), which is likely due to them being very flexible, while taking the mean result keeps the variance down. The performance of the models are only slightly higher than that of the most naive classifier, which has an accuracy of 82,4\% on the testing data. This is likely due to the fact that migration of stars is dependent on other factors, which probably are local to the formation place of the stars and therefore non-predictable from current data. It is also worth noticing that our models are trained on simulation data and may not, although unlikely, be unusable in the field. They should be seen as a statistical tool, to point researchers in different directions. They could be improved by trying even more different models and training the models on the full dataset.